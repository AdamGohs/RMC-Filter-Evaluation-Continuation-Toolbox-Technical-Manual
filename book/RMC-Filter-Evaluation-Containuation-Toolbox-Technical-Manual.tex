% Options for packages loaded elsewhere
\PassOptionsToPackage{unicode}{hyperref}
\PassOptionsToPackage{hyphens}{url}
%
\documentclass[
]{book}
\usepackage{amsmath,amssymb}
\usepackage{iftex}
\ifPDFTeX
  \usepackage[T1]{fontenc}
  \usepackage[utf8]{inputenc}
  \usepackage{textcomp} % provide euro and other symbols
\else % if luatex or xetex
  \usepackage{unicode-math} % this also loads fontspec
  \defaultfontfeatures{Scale=MatchLowercase}
  \defaultfontfeatures[\rmfamily]{Ligatures=TeX,Scale=1}
\fi
\usepackage{lmodern}
\ifPDFTeX\else
  % xetex/luatex font selection
\fi
% Use upquote if available, for straight quotes in verbatim environments
\IfFileExists{upquote.sty}{\usepackage{upquote}}{}
\IfFileExists{microtype.sty}{% use microtype if available
  \usepackage[]{microtype}
  \UseMicrotypeSet[protrusion]{basicmath} % disable protrusion for tt fonts
}{}
\makeatletter
\@ifundefined{KOMAClassName}{% if non-KOMA class
  \IfFileExists{parskip.sty}{%
    \usepackage{parskip}
  }{% else
    \setlength{\parindent}{0pt}
    \setlength{\parskip}{6pt plus 2pt minus 1pt}}
}{% if KOMA class
  \KOMAoptions{parskip=half}}
\makeatother
\usepackage{xcolor}
\usepackage{longtable,booktabs,array}
\usepackage{calc} % for calculating minipage widths
% Correct order of tables after \paragraph or \subparagraph
\usepackage{etoolbox}
\makeatletter
\patchcmd\longtable{\par}{\if@noskipsec\mbox{}\fi\par}{}{}
\makeatother
% Allow footnotes in longtable head/foot
\IfFileExists{footnotehyper.sty}{\usepackage{footnotehyper}}{\usepackage{footnote}}
\makesavenoteenv{longtable}
\usepackage{graphicx}
\makeatletter
\def\maxwidth{\ifdim\Gin@nat@width>\linewidth\linewidth\else\Gin@nat@width\fi}
\def\maxheight{\ifdim\Gin@nat@height>\textheight\textheight\else\Gin@nat@height\fi}
\makeatother
% Scale images if necessary, so that they will not overflow the page
% margins by default, and it is still possible to overwrite the defaults
% using explicit options in \includegraphics[width, height, ...]{}
\setkeys{Gin}{width=\maxwidth,height=\maxheight,keepaspectratio}
% Set default figure placement to htbp
\makeatletter
\def\fps@figure{htbp}
\makeatother
\setlength{\emergencystretch}{3em} % prevent overfull lines
\providecommand{\tightlist}{%
  \setlength{\itemsep}{0pt}\setlength{\parskip}{0pt}}
\setcounter{secnumdepth}{5}
\usepackage{booktabs}
\usepackage{amsthm}
\usepackage{natbib} % Required for bibliography
\usepackage{csquotes} % Provides advanced quoting and citation features

% Define the citation-note environment if it's required
\newenvironment{citation-note}
    {\begin{quote}\itshape}
    {\end{quote}}

\makeatletter
\def\thm@space@setup{%
  \thm@preskip=8pt plus 2pt minus 4pt
  \thm@postskip=\thm@preskip
}
\makeatother
\ifLuaTeX
  \usepackage{selnolig}  % disable illegal ligatures
\fi
\usepackage[]{natbib}
\bibliographystyle{plainnat}
\usepackage{bookmark}
\IfFileExists{xurl.sty}{\usepackage{xurl}}{} % add URL line breaks if available
\urlstyle{same}
\hypersetup{
  pdftitle={RMC Filter Evaluation (Continuation) Toolbox Technical Manual},
  pdfauthor={Adam C. Gohs},
  hidelinks,
  pdfcreator={LaTeX via pandoc}}

\title{RMC Filter Evaluation (Continuation) Toolbox Technical Manual}
\author{Adam C. Gohs}
\date{Last updated: 2025-01-29}

\begin{document}
\maketitle

{
\setcounter{tocdepth}{1}
\tableofcontents
}
\chapter*{Preface}\label{preface}
\addcontentsline{toc}{chapter}{Preface}

The Risk Management Center (RMC) of the U.S. Army Corps of Engineers (USACE) has developed a suite of Microsoft Excel spreadsheets to support risk assessments for dam and levee safety. Each analysis suite is composed of multiple toolboxes (Microsoft Excel workbooks), and each toolbox contains multiple spreadsheet tools or calculation worksheets (Microsoft Excel worksheets). The RMC Filter Evaluation (Continuation) Toolbox is part of the RMC Internal Erosion Suite.

The spreadsheet tools contained in this toolbox assess the particle retention and permeability criteria for filter materials It includes modern no erosion filter design criteria, Foster and Fell (2001) method for filters that do not meet modern filter design criteria, and the Fell and Foster (2023) method for ``continuing erosion'' condition for constricted exits.

\subsection*{How to Cite This Guide}\label{how-to-cite-this-guide}
\addcontentsline{toc}{subsection}{How to Cite This Guide}

If you would like to cite this User's Guide, please use the following format:

\begin{citation-note}
A. C. Gohs, \emph{RMC Filter Evaluation (Continuation) Toolbox Technical
Manual}, Lakewood, CO: U.S. Army Corps of Engineers, Risk Management
Center, 2024. Accessed on {\textless enter current date
here\textgreater{}}.
\end{citation-note}

  \bibliography{book.bib}

\end{document}
